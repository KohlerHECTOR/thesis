\chapter*{Preliminary Concepts}

\section{What is Sequential Decision Making?}
In this manuscript we are interested in sequential decision making. Humans engage in sequential decision making in all aspects of life. In medicine, doctors have to decide when to use chemotherapy next based on the patient's current health in order to heal (cite). In agriculture, agronomists have to decide when to fertilize next based on the current soil and weather conditions in order to maximize plant growth (cite). In automotive, the auto-pilot system has to decide how to steer the wheel next based on lidar sensors in order to maintain a safe trajectory (cite). In video games, a bot decides what attack to throw next based on the player's and its own state in order to provide the best entertainment (cite).
Those sequential decision making processes exhibits key similarities: an agent takes actions based on some current information to achieve some goal.
As computer scientists, we ought to design algorithms that can help humans during those sequential decision making processes. For example, a doctor could benefit from a computer program (cite) that would recommend the ``best'' treatment given the patient's state. 
Algorithms presented in this thesis use the formalism of Markov decision processes that we present next to model sequential decision making.
\subsection{Markov decision processes/problems}
(figure)
Markov decision processes (MDPs) were first introduced in the 1950s by Richard Bellman (cite). Informally, an MDP models how an agent acts over time to achieve its goal. At every timestep, the agent observes its current state, e.g. a patient weight and tumor size, and takes an action, e.g. injects a certain amount of chemotherapy. When doing a certain action in a certain state, the agent gets a reward that helps it evaluate the quality of its action with respect to its goal, e.g., the tumor size decrease when the agent has to cure cancer. Finally, the agent is provided with a new state, e.g. the updated patient state, and repeats this process over time. Following Martin L. Puterman's book on MDPs (cite), we formally define as follows.
\begin{definition}[Markov decision process] An MDP is a tuple $\mathcal{M} = \langle S, A, R, T, T_0 \rangle$ where:
\begin{itemize}
\item $S$ is a finite set of states $s \in \mathbb{R}^n$ representing all possible configurations of the environment.
\item $A$ is a finite set of actions $a \in \mathbb{Z}^m$ available to the agent.
\item $R: S \times A \rightarrow \mathbb{R}$ is the reward function that assigns a real-valued reward to each state-action pair.
\item $T: S \times A \rightarrow \Delta(S)$ is the transition function that maps state-action pairs to probability distributions over next states, where $\Delta(S)$ denotes the probability simplex over $S$.
\item $T_0 \in \Delta(S)$ is the initial distribution over states.
\end{itemize}
\end{definition}
Now we can also model the ``goal'' of the agent. Informally, the goal of an agent is to behave such that it gets as much reward as it can over time. For example, in the cancer treatment case, the best reward the agent can get is to completely get rid of the patient's tumor after some time. Furthermore, we want our agent to prefer behaviour that gets rid of the patient's tumor as fast as possible. We can formally model the agent's goal as an optimization problem as follows. %talk about alignment?
\begin{definition}[Markov decision problem] Given an MDP $\mathcal{M}=\langle S, A, R, T, T_0 \rangle$, the goal of an agent following policy $\pi: S \rightarrow A$ is to maximize the expected discounted sum of rewards:
$$J(\pi) = \mathbb{E}\left[\sum_{t=0}^{\infty} \gamma^t R(s_t, a_t) \mid s_0 \sim T_0, a_t = \pi(s_t), s_{t+1} \sim T(s_t, a_t)\right]$$
where $\gamma \in (0,1)$ is the discount factor that controls the trade-off between immediate and future rewards.
\end{definition}
Hence, algorithms presented in this manuscript aim to find solutions to Markov decision problems, i.e. the optimal policy: $\pi^\star =\underset{\pi}{\operatorname{argmax}}J(\pi)$
For the rest of this text, we will use an abuse of notation and denote both a Markov decision process and the associated Markov decision problem by MDP.
\subsection{Exact solutions for Markov decision problems}
It is possible to compute the exact optimal policy $\pi^\star$ using dynamic programming (cite). Indeed, one can leverage the Markov property to find for all states the best action to take based on the reward of upcoming states.
\begin{definition}[Value of a state] The value of a state $s\in S$ under policy $\pi$ is the expected discounted sum of rewards starting from state $s$ and following policy $\pi$:
    $$V^\pi(s) = \mathbb{E}\left[\sum_{t=0}^{\infty} \gamma^t R(s_t, a_t) \mid s_0 = s, a_t = \pi(s_t), s_{t+1} \sim T(s_t, a_t)\right]$$
    Applying the Markov property gives a recursive definition of the value of $s$ under policy $\pi$:
    $$V^\pi(s) = R(s,\pi(s)) + \gamma \sum_{s' \in S} T^{s'}(s,\pi(s))V^\pi(s')$$
    where $T^{s'}(s,\pi(s))$ is the probability of transitioning to state $s'$ when taking action $\pi(s)$ in state $s$.
\end{definition}
\begin{definition}[Optimal value of a state] The optimal value of a state $s\in S$, $V^\star(s)$, is the value of state $s$ when following the optimal policy: $V^{\pi^{\star}}(s)$.
    $$V^{\star}(s) = V^{\pi^{\star}}(s) = \underset{\pi}{\max}\left[J(\pi)\right]$$
\end{definition}
Hence, the algorithms we study in the thesis can also be seen as solving the problem: $\pi^{\star} = \underset{\pi}{\operatorname{argmax}}\mathbb{E}\left[V^{\pi}(s_0)|s_0\sim T_0 \right]$. The well-known Value Iteration algorithm (algorithm) solves this problem exactly (cite). 

More realistically, neither the transition kernel $T$ nor the reward function $R$ of the MDP are known, e.g., the doctor can't \textbf{know} how the tumor and the patient health will change after a dose of chemotherapy, it can only \textbf{observe} the change. This distinction between the information available to the agent is paralleled with the distinction between dynamic programming and reinforcement learning (RL) that we describe next. 
\subsection{Reinforcement learning of approximate solutions to MDPs}
(figrure)
Reinforcement learning algorithms popularized by Richard Sutton (figure) (cite) don't \textbf{compute} an optimal policy but rather \textbf{learn} an approximate one based on sequences of observations ${(s_t, a_t, r_t, s_{t+1})}_t$.
RL algorithms usually fall into two categories: value-based (cite) and policy gradient (cite). The first group of RL algorithms computes an appromiation of $V^{\star}$ using temporal different learning (algorithms) (cite). The other class of RL algorithms leverage the policy gradient theorem (algorithm) (cite) to approximate $\pi^{\star}$.
Both class of algorithms are known to converge to the optimal value or policy under some conditions (cite) and have known great successes in real-world appilcations (cite).
The books from Puterman, Bertsekas, Sutton and Barto, offer a great overview of MDPs and algorithm to solve them.
There are many other ways to learn policies such as simple random search (cite) or model-based reinforcement learning. However, not many algorithms consider the learning of policies that can be easily understood by humans which we discuss next and that is the core of this manuscript.
\section{Why do we need Interpretable Policies for Sequential Decision Making?} 
(figure)
Interpretability is a crucial topic in modern science. While computer programs trained with machine learning algorithms have become more and more performing and made their way into our society, they are often black-box: their outputs, e.g., the animals on the image (cite), tokamak control (cite), or even the abstract of your next article (cite), are computed with operations that are too complex for humans to fully understand.
Indeed most of recent machine learning breakthroughs are obtained by training very large neural networks (cite) which are black-box by definition since they apply composition of scalar products on the inputs.
Originally, the etymology of ``interpretability'' is the Latin ``interpretabilis'' meaning ``that can be understood and explained''.
According to the Oxford English dictionary, the first recorded use of the english word ``interpretability'' dates back to 1854 when the british logician George Boole (figure) described the addition of concepts:

\begin{displaycquote}[pp.~48]{(cite)}
I would remark in the first place that the generality of a method in Logic
must very much depend upon the generality of its elementary processes and laws.
We have, for instance, in the previous sections of this work investigated, among
other things, the laws of that logical process of addition which is symbolized by
the sign +. Now those laws have been determined from the study of instances,
in all of which it has been a necessary condition, that the classes or things added
together in thought should be mutually exclusive. The expression x + y seems
indeed uninterpretable, unless it be assumed that the things represented by x
and the things represented by y are entirely separate; that they embrace no
individuals in common. And conditions analogous to this have been involved
in those acts of conception from the study of which the laws of the other
symbolical operations have been ascertained. The question then arises, whether
it is necessary to restrict the application of these symbolical laws and processes
by the same conditions of interpretability under which the knowledge of them
was obtained. If such restriction is necessary, it is manifest that no such thing
as a general method in Logic is possible. On the other hand, if such restriction
is unnecessary, in what light are we to contemplate processes which appear to
be uninterpretable in that sphere of thought which they are designed to aid?
\end{displaycquote}
What is remarkable is that the supposedly first recored occurrence of ``interpretability'' was in the context of (pre-)computer science. Boole asked: \textit{when can we meaningfully apply formal mathematical operations beyond the specific conditions under which we understand them?}
In Boole's era, the concern was whether logical operations like addition could be applied outside their original interpretable contexts—where symbols and their sum represent concepts that humans can understand, e.g. red + apples = red apples. Today, we face an analogous dilemma with machine learning algorithms: neural networks learn complex un-intellegible combinations of inputs (representations), but we often deploy them in contexts where operations should be unerstood by humans, e.g., in medicine. 

Circling back to our cancer treatment example, we would ideally want doctors to have access to computer programs that can recommend ``good'' treatments and which operations can be understood. Those two aspects of machine learning models--performance and interpretability--often compromise; highly performing models like neural networks are often less interpretable and vice-versa (cite).

Interestingly, one of the key challenges of doing research in interpretability is the lack of formalism; there is no definition of what is an interpretable compute program such as an MDP policy. Throughout this manuscript we make the hypothesis that interpretability is the (space and time) complexity of a program and hence mostly focus on decision trees (low complexity) (cite) and neural networks (high complexity).  

% Then talk about the latin sense. 
% Then talkabout no formal def but books, articles, workshops, ...
\subsection{What are existing approaches for interpretable sequential decision making?}
(figure)

\subsection{What are decision tree policies?}
(figure)

\subsection{Why are decision tree policies harder to learn than decision tree classifiers?}

\section{Outline of the Thesis}
(figure)