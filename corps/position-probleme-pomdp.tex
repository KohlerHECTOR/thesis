\chapter{A Decision Tree Policy for an MDP is a Policy for some Partially Observable MDP}
\epigraphhead[30]{\selectlanguage{english}\epigraph{I have not failed. I've
    just found 10.000 ways that won't work.}{Thomas A. Edison}}


\section{How to Learn a Decision Tree Policy for an MDP?}
\subsection{Imitation}
\subsection{Soft Trees}
\subsection{Iterative Bounding MDPs}

\section{How to solve Iterative Bounding MDPs?}
\subsection{Asymmetric Reinforcement Learning}
\subsection{Learning a decision tree policy is solving a POMDP}

\section{Is it hard to properly learn a Decision Tree Policy for an MDP?}
\subsection{POMDPs are way harder to solve than MDPs}
\subsection{Memoryless approaches to solve POMDPs seem uneffective}
% We consider a 2×2 grid world Markov Decision Process (MDP) defined as follows:
% \begin{itemize}
%     \item \textbf{States}: Four cells labeled $S_0$, $S_1$, $S_2$, and $G$ (goal state) arranged in a 2×2 grid.
%     \item \textbf{Actions}: At each state, the agent can move right ($\rightarrow$) or down ($\downarrow$) up ($\uparrow$) or left ($\leftarrow$).
%     \item \textbf{Transitions}: Movements are deterministic, following the direction of the chosen action. Actions that would lead outside the grid leave the agent in the same state.
%     \item \textbf{Rewards}: All transitions yield a reward of 0, except for any action taken from the goal state $G$, which yields a reward of 1.
%     \item \textbf{Objective}: Maximize the expected discounted cumulative reward.
% \end{itemize}

% \begin{figure}[ht]
% \centering
% \begin{tikzpicture}[scale=1.5]
%     % Draw the grid cells
%     \draw (0,0) grid (2,2);
    
%     % Add ticks on axes
%     \foreach \x in {0,1,2}
%         \node[below] at (\x,0) {$\x$};
%     \foreach \y in {0,1,2}
%         \node[left] at (0,\y) {$\y$};
    
%     \node[left] at (-0.5, 1) {$y$};
%     \node[below] at (1, -0.5) {$x$};
    
%     % Label cells
%     \node at (0.5,0.7) {$S_0$};
%     \node at (0.5,0.4) {$\times$};

%     \node at (0.5,1.7) {$S_1$};
%     \node at (0.5,1.4) {$\times$};

%     \node at (1.5,1.7) {$S_2$};
%     \node at (1.5,1.4) {$\times$};

    
%     % Goal state in bottom right with double border
%     \draw[line width=2pt] (1,0) rectangle (2,1);
%     \node at (1.5,0.7) {$G$};
%     \node at (1.5,0.4) {$\star$};

    
% \end{tikzpicture}
% \caption{The 2×2 grid world environment with states $S_0$, $S_1$, $S_2$, and goal state $G$.}\label{fig:grid-world}
% \end{figure}


% \begin{figure}[ht]
% \centering
% \begin{tikzpicture}[scale=1.5]
%     % Draw the grid cells
%     \draw (0,0) grid (2,2);
    
%     % Add ticks on axes
%     \foreach \x in {0,1,2}
%         \node[below] at (\x,0) {$\x$};
%     \foreach \y in {0,1,2}
%         \node[left] at (0,\y) {$\y$};
    
%     \node[left] at (-0.5, 1) {$y$};
%     \node[below] at (1, -0.5) {$x$};
    
%     % Label cells
%     \node at (0.5,0.5) {$\times$};
%     \node at (0.6,0.48) {$\rightarrow$};

%     \node at (0.5,1.5) {$\times$};
%     \node at (0.5,1.4) {$\downarrow$};

%     \node at (1.5,1.5) {$\times$};
%     \node at (1.5,1.4) {$\downarrow$};
    
%     % Goal state in bottom right with double border
%     \draw[line width=2pt] (1,0) rectangle (2,1);
%     \node at (1.5,0.5) {$\star$};
%     \node at (1.6,0.48) {$\rightarrow$};

    
% \end{tikzpicture}
% \caption{The optimal tabular policy for the grid world.}\label{fig:optimal-policy}
% \end{figure}

% \begin{figure}[ht]
% \centering
% \begin{tikzpicture}[
%     scale=1.2,
%     decision/.style={circle, draw, text width=1.5em, text centered, minimum height=2.5em},
%     action/.style={rectangle, draw, text width=2em, text centered, rounded corners}
% ]
%     % Decision node
%     \node[decision] (decide) at (0,0) {$y<1$};
    
%     % Action nodes
%     \node[action] (right) at (1.5,-1.5) {$\downarrow$};
%     \node[action] (left) at (-1.5,-1.5) {$\rightarrow$};
    
%     % Connections and labels
%     \draw[->] (decide) -- node[right] {no} (right);
%     \draw[->] (decide) -- node[left] {yes} (left);
% \end{tikzpicture}
% \caption{If $y < 1$, move right; otherwise, move down.}\label{fig:dt-simple}
% \end{figure}

% \begin{figure}[ht]
% \centering
% \begin{tikzpicture}[
%     scale=1.2,
%     decision/.style={circle, draw, text width=1.5em, text centered, minimum height=2.5em},
%     action/.style={rectangle, draw, text width=2em, text centered, rounded corners}
% ]
%     % Decision node
%     \node[decision] (decide) at (0,0) {$x<1$};
    
%     % Action nodes
%     \node[action] (right) at (1.5,-1.5) {$\downarrow$};
%     \node[decision] (decide2) at (-1.5,-1.5) {$y<1$};
%     \node[action] (right2) at (0,-3) {$\downarrow$};
%     \node[action] (left2) at (-3,-3) {$\rightarrow$};
    
%     % Connections and labels
%     \draw[->] (decide) -- node[right] {no} (right);
%     \draw[->] (decide) -- node[left] {yes} (decide2);
%     \draw[->] (decide2) -- node[left] {yes} (left2);
%     \draw[->] (decide2) -- node[right] {no} (right2);
% \end{tikzpicture}
% \caption{A redundant decision tree.}\label{fig:dt-complex}
% \end{figure}

% \begin{figure}[ht]
% \centering
% \begin{tikzpicture}[
%     scale=1.2,
%     decision/.style={circle, draw, text width=1.5em, text centered, minimum height=2.5em},
%     action/.style={rectangle, draw, text width=2em, text centered, rounded corners}
% ]
%     % Decision node
%     \node[action] (decide) at (0,0) {$\rightarrow$};
% \end{tikzpicture}
% \caption{A trivial single-action policy that always selects the rightward action, regardless of the agent's position.}\label{fig:dt-trivial}
% \end{figure}


% \begin{tikzpicture}
%     \begin{axis}[
%         width=10cm,
%         height=6cm,
%         xlabel={$\gamma$},
%         ylabel={$\zeta$},
%         title={Bounds on $\zeta$},
%         legend pos=north west,
%         grid=major,
%         grid style={gray!30},
%     ]
    
%     % Lower bound from data file
%     \addplot[blue, thick] table[x=gamma, y=lower_bound] {zeta_bounds_data.dat};
    
%     % Upper bound from data file  
%     \addplot[red, thick] table[x=gamma, y=upper_bound] {zeta_bounds_data.dat};
    
%     % Optional: fill between if you want the shaded region
%     % \addplot[green!20, forget plot] fill between[of=A and B] restrict z to domain*={zeta_bounds_data.dat:valid==1};
    
%     \legend{Lower bound, Upper bound}
%     \end{axis}
%     \end{tikzpicture}

% Figure illustrates how these bounds vary with the discount factor $\gamma$. For smaller values of $\gamma$, the bound from $J(T_0) \leq J(T_1)$ dominates, while for larger values of $\gamma$, the bound from $J(T_2) \leq J(T_1)$ becomes more restrictive.



