% Auteur de la thèse : prénom (1er argument obligatoire), nom (2e argument
% obligatoire) et éventuel courriel (argument optionnel). Les éventuels accents
% devront figurer et le nom /ne/ doit /pas/ être saisi en capitales
\author[hector.kohler@inria.fr]{Hector}{Kohler}
%
% Titre de la thèse dans la langue principale (argument obligatoire) et dans la
% langue secondaire (argument optionnel)
\title[Decision Trees and Sequential Decision Making]{Arbres de D\'ecision et Prise de D\'ecisions S\'equentielle}
%
% (Facultatif) Sous-titre de la thèse dans la langue principale (argument
% obligatoire) et dans la langue secondaire (argument optionnel)
% \subtitle[Chaos' Laugh]{Le rire du chaos}
%
% Champ disciplinaire dans la langue principale (argument obligatoire) et dans
% la langue secondaire (argument optionnel)
\academicfield[Computer Science]{Informatique}
%
% (Facultatif) Spécialité dans la langue principale (argument obligatoire) et
% dans la langue secondaire (argument optionnel)
\speciality[Computer Science and Applications]{Informatique et Applications}
%
% Date de la soutenance, au format {jour}{mois}{année} donnés sous forme de
% nombres
\date{1}{12}{2025}
%
% (Facultatif) Date de la soumission, au format {jour}{mois}{année} donnés sous
% forme de nombres
\submissiondate{1}{9}{2025}
%
%
% Nom (argument obligatoire) de l'institut (principal en cas de cotutelle)
\institute[logo=images/ulille,url=https://www.univ-lille.fr/]{Universit\'e de Lille}
%
% (Facultatif) En cas de cotutelle (normalement, seulement dans le cas de
% cotutelle internationale), nom (argument obligatoire) du second institut
\coinstitute[logo=images/inr_logo_rouge.png,url=http://www.inria.fr/]{Inria}
%
% (Facultatif) Nom (argument obligatoire) de l'école doctorale
\doctoralschool[url=https://edmadis.univ-lille.fr/]{\'Ecole Gradu\'ee MADIS-631}
%
% Nom (1er argument obligatoire) et adresse (2e argument obligatoire) du
% laboratoire (ou de l'unité) où la thèse a été préparée, à utiliser /autant de
% fois que nécessaire/
\laboratory[
logo=images/cristal,
logoheight=1.25cm,
url=https://www.cristal.univ-lille.fr/
]{Centre de Recherche en Informatique, Signal et Automatique de Lille}{%
  Université de Lille-Campus scientifique \\
  Bâtiment ESPRIT\\
  Avenue Henri Poincaré\\
  59655 Villeneuve d'Ascq}
%
% Directeur(s) de thèse et membres du jury, saisis au moyen des commandes
% \supervisor, \cosupervisor, \comonitor, \referee, \committeepresident,
% \examiner, \guest, à utiliser /autant de fois que nécessaire/ et /seulement
% si nécessaire/. Toutes basées sur le même modèle, ces commandes ont
% 2 arguments obligatoires, successivement les prénom et nom de chaque
% personne. Si besoin est, on peut apporter certaines précisions en argument
% optionnel, essentiellement au moyen des clés suivantes :
% - « professor », « seniorresearcher », « associateprofessor »,
%   « associateprofessor* », « juniorresearcher », « juniorresearcher* » (qui
%   peuvent ne pas prendre de valeur) pour stipuler le corps auquel appartient
%   la personne ;
% - « affiliation » pour stipuler l'institut auquel est affiliée la personne ;
% - « female » pour stipuler que la personne est une femme pour que certains
%   mots clés soient accordés en genre.
%
\supervisor[professor,affiliation=Universit\'e de Lille]{Philippe}{Preux}
\cosupervisor[juniorresearcher,affiliation=Inria]{Riad}{Akrour}
\referee[juniorresearcher,affiliation=Universit\'e de Lorraine, Inria]{Olivier}{Buffet}
\referee[mcf,affiliation=Sorbonne Universit\'es]{Aur\'elie}{Beynier}
\committeepresident[professor,affiliation=Universit\'e de Lorraine]{Lydia}{Boudjeloud-Assala}
\examiner[mcf,affiliation=University of Pennsylvania]{Osbert}{Bastani}
\guest{Sonali}{Parbhoo}
%
% (Facultatif) Mention du numéro d'ordre de la thèse (s'il est connu, ce numéro
% est à spécifier en argument optionnel)
\ordernumber[42]
%
% Préparation des mots clés dans la langue principale (1er argument) et dans la
% langue secondaire (2e argument)
%%%%%%%%%%%%%%%%%%%%%%%%%%%%%%%%%%%%%%%%%%%%%%%%%%%%%%%%%%%%%%%%%%%%%%%%%%%%%%%
\keywords{apprentissage par renforcement, arbres de d\'ecision, interpr\'etabilit\'e, m\'ethodologie}{reinforcement learning, deicision trees, interpretability, methodology}

