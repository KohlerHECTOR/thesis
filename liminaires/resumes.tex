% Résumés (de 1700 caractères maximum, espaces compris) dans la
% langue principale (1re occurrence de l'environnement « abstract »)
% et, facultativement, dans la langue secondaire (2e occurrence de
% l'environnement « abstract »)
\begin{abstract}
  Dans cette thèse de doctorat, nous étudions des algorithmes d'apprentissage d'arbres de décision pour la classification et la prise de décision séquentielle. Les arbres de décision sont interprétables car les humains peuvent lire les opérations de l'arbre de décision depuis la racine jusqu'aux feuilles. Cela fait des arbres de décision le modèle de référence lorsque la vérification humaine est requise, comme dans les applications médicales. Cependant, les arbres de décision ne sont pas différentiables, ce qui les rend difficiles à optimiser, contrairement aux réseaux neuronaux qui peuvent être entraînés efficacement avec la descente de gradient. Les approches existantes d'apprentissage par renforcement interprétables apprennent généralement des arbres souples (non interprétables en l'état) ou sont ad hoc (entraînent un réseau neuronal puis entraînent un arbre à imiter le réseau). Cet apprentissage d'arbre indirect ne garantit pas de trouver des bonnes solutions pour le problème initial.

  Dans la première partie de ce manuscrit, nous visons à apprendre directement des arbres de décision pour un processus de décision Markovien avec de l'apprentissage par renforcement. En pratique, nous montrons que cela revient à résoudre un problème de décision Markovien partiellement observable (PDMPO). La plupart des algorithmes d'apprentissage par renforcement existants ne sont pas adaptés aux PDMPOs. Ce parallèle entre l'apprentissage des arbres de décision et la résolution des PDMPOs nous aide à comprendre pourquoi, dans la pratique, il est souvent plus facile d'obtenir une politique experte non interprétable (un réseau neuronal) puis de la distiller en un arbre plutôt que d'apprendre l'arbre de décision à partir de zéro.

  La deuxième contribution de ce travail découle de l'observation selon laquelle la recherche d'un classifieur (ou régresseur) arbre de décision est une instance complètement observable du problème précédent. Nous formulons donc l'induction d'arbres de décision comme la résolution d'un problème de décision Markovien et proposons un nouvel algorithme de pointe qui peut être entraîné à partir de données d'exemple supervisés et qui généralise bien à des données nouvelles.

  Les travaux des parties précédentes reposent sur l'hypothèse que les arbres de décision sont un modèle interprétable que les humains peuvent utiliser dans des applications sensibles. Mais est-ce vraiment le cas ? Dans la dernière partie de cette thèse, nous tentons de répondre à des questions plus générales sur l'interprétabilité : pouvons-nous mesurer l'interprétabilité sans intervention humaine ? Et les arbres de décision sont-ils vraiment plus interprétables que les réseaux neuronaux ?

\end{abstract}
\begin{abstract}
  In this Ph.D. thesis, we study algorithms to learn decision trees for classification and sequential decision-making. Decision trees are interpretable because humans can read through the decision tree computations from the root to the leaves. This makes decision trees the go-to model when human verification is required like in medicine applications. However, decision trees are non-differentiable making them hard to optimize unlike neural networks that can be trained efficiently with gradient descent. Existing interpretable reinforcement learning (RL) approaches usually learn soft trees (non-interpretable as is) or are ad-hoc (train a neural network then fit a tree to it) potentially missing better solutions.

  In the first part of this manuscript, we aim to directly learn decision trees for a Markov decision process with reinforcement learning. In practice we show that this amounts to solving a partially observable Markov decision problem (POMDP). Most existing RL algorithms are not suited for POMDPs. This parallel between decision tree learning with RL and POMDPs solving help us understand why in practice it is often easier to obtain a non-interpretable expert policy--a neural network--and then distill it into a tree rather than learning the decision tree from scratch.

  The second contribution from this work arose from the observation that looking for a decision tree classifier (or regressor) is a fully observable instance of the above problem. We thus formulate decision tree induction as solving a Markov decision problem and propose a new state-of-the-art algorithm that can be trained with supervised example data and generalizes well to unseen data.

  Work from the previous parts rely on the hypothesis that decision trees are indeed an interpretable model that humans can use in sensitive applications. But is it really the case? In the last part of this thesis, we attempt to answer some more general questions about interpretability: can we measure interpretability without humans? And are decision trees really more interpretable than neural networks?
\end{abstract}
%
% Production de la page de résumés
\makeabstract{}

