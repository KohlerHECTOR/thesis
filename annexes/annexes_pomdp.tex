\chapter{Appendix for part \ref{part1}}
\label{chap-app-pomdp}
In this appendix, we detail the experimental setup used to reproduce~\cite[table 1]{topin2021iterative} in chapter~\ref{sec:topin}.
We also provide details of the calculation of the optimization objective values used in figure~\ref{fig:irl-objectives}.
Finally, we provide additional experimental details for chapter~\ref{sec:pomdp}.
\section{Details of the reproducibility study}

\subsection{IBMDP formulation}\label{sec:ibmdp-paper}
Given a base MDP $\mathcal{M}\langle S, A, R, T, T_0\rangle$ (cf. definition~\ref{def:mdp}), in order to define an IBMDP $\mathcal{M}_{IB}\langle S\times O, A\cup A_{info}, (R, \zeta),( T, T_0, T_{info})\rangle$ (cf. definition~\ref{def:ibmdp}), the user needs to provide the set of information gathering actions $A_{info}$ and the reward $\zeta$ for taking those.
Authors of~\cite{topin2021iterative} propose to parametrize the set of IGAs with $j \times q$ actions $\langle j, v_k \rangle$ with $v_k$ depending on the current observation $\boldsymbol{o}_t=(L'_1, U'_1, \dots, L'_j, U'_j, \dots, L'_p, U'_p)$: $v_k = \frac{k(U'_j - L'_j)}{q+1}$.
This parametric IGAs space keeps the discrete IBMDP action space at a reasonable size while providing a learning algorithm with varied IGAs to try.

For example, if we define an IBMDP with $q=3$ for the grid world from Example~\ref{example:grid}, the grid world action space is augmented with six IGAs. 
At $t=0$, recall that $\boldsymbol{o}_0=(0, 2, 0, 2)$, so if an IGA is taken, e.g. $\langle 2, v_2\rangle$, the effective IGA is $\langle j, v_2=\frac{k(2-0)}{3+1}\rangle = \langle 1, 2 \rangle$ which in turn effectively corresponds to an internal decision tree node $y \leq 1$.
If the current state $y$-feature value is $0.5$, then the next observation at $t=1$ is $\boldsymbol{o}_1=(0, 2, 0, 1)$. At $t=2$ if $a_t=\langle 2, v_2 \rangle$ again, it would be effectively $\langle j, v_2=\frac{k(1-0)}{3+1}\rangle = \langle 2, 0.5 \rangle$. 
This would give the next observation at $t=2$ $\boldsymbol{o}_2=(0, 2, 0, 0.5)$ and so on. 

Furthermore, author propose to regularize the learned decision tree policy with a maximum depth parameter $D$.
Unfortunately, authors did not describe how they implemented the depth control in their work, hence we have to try different approaches to reproduce their results.

To control the tree depth during learning in the IBMDP, we can either give negative reward for taking $D$ IGAs in a row, or terminate the trajectory.
In practice, we could also have a state-dependent action space such that taking an IGA is not allowed after taking $D$ IGAs in a row.
The latter approach--sometimes called action masking--is not compatible with the definition of an MDP (cf. definition~\ref{def:mdp}) in which all actions are available in all states. 
To apply the penalization approaches, one can extend the MDP states to keep track of the current tree depth.
Similarly, the termination approach requires a transition function that depends on the current tree depth.

We actually find that when $q+1$, the parameter that defines threshold values in decision tree policy nodes (cf. definition~\ref{def:ibmdp}), is a prime number, then as a direct consequence of the \textit{Chinese Remainder Theorem}\footnote{\url{https://en.wikipedia.org/wiki/Chinese_remainder_theorem}}, the current tree depth is directly encoded in the current observation $\boldsymbol{o}_t$. 
Hence, when $q+1$ is prime, we can control the depth through either transitions or rewards without tracking the tree depth.

\begin{table}[H]
    \centering
    \caption{IBMDP hyperparameters. We try 12 different IBMDPs. In green we highlight the hyperparameters from the original paper and in red we highlight the hyperparameter names for which author do not give information.}\label{tab:ibmdp-params}
    \begin{tabular}{ll}
    \toprule
    \textbf{Hyperparameter} & \textbf{Values}\\
    \midrule
    Discount factor $\gamma$ & \textcolor{green}{1} \\
    Information gathering actions parameter $q$ & 2, \textcolor{green}{3} \\
    Information gathering actions rewards $\zeta$ & \textcolor{green}{-0.01}, 0.01 \\
    \textcolor{red}{Depth control} & Done signal, negative reward, none \\ 
    \bottomrule
    \end{tabular}
    \end{table}

\subsection{Modified RL algorithmic details}
\begin{algorithm}
    \KwData{IBMDP $\mathcal{M}_{IB}\langle S\times O, A\cup A_{info}, (R, \zeta),( T, T_0, T_{info})\rangle$, learning rate $\alpha$, policy parameters $\theta$, clipping parameter $\epsilon$, value function parameters $\phi$}
    \KwResult{Partially observable stochastic policy $\pi_{po_\theta}$}
    Initialize policy parameters $\theta$ and value function parameters $\phi$ \\
    \For{each episode}{
        Generate trajectory $\tau = (\boldsymbol{s}_0, \boldsymbol{o}_0, a_0, r_0, \boldsymbol{s}_1, \boldsymbol{o}_1, a_1, r_1, \ldots)$ following $\pi_\theta$ \\
        \For{each timestep $t$ in trajectory}{
            $G_t \leftarrow \sum_{k=t}^{T} \gamma^{k-t} r_k$ \Comment{// Compute return}
            $A_t \leftarrow G_t - V_\phi(\boldsymbol{s}_t)$ \Comment{// Compute advantage}
            $r_t(\theta) \leftarrow \frac{\pi_{po_\theta}(a_t|\boldsymbol{o}_t)}{\pi_{po_\theta}_{old}(a_t|\boldsymbol{o}_t)}$ \Comment{// Compute probability ratio}
            $L^{CLIP}_t \leftarrow \min(r_t(\theta) A_t, \text{clip}(r_t(\theta), 1-\epsilon, 1+\epsilon) A_t)$ \Comment{// Clipped objective}
            $\theta \leftarrow \theta + \alpha \nabla_\theta L^{CLIP}_t$ \Comment{// Policy update}
            $\phi \leftarrow \phi + \alpha \nabla_\phi (G_t - V_\phi(\boldsymbol{s}_t))^2$ \Comment{// Value function update}
        }
        $\theta_{old} \leftarrow \theta$ \Comment{// Update old policy}
    }
    \caption{Modified Proximal Policy Optimization}\label{alg:mod-ppo}
\end{algorithm}

\begin{table}[H]
    \centering
    \caption{(Modified) DQN trained on $10^6$ timesteps. This gives four different instantiation of (modified) DQN. Hyperparameters not mentioned are stable-baselines3 default. In green we highlight the hyperparameters from the original paper and in red we highlight the hyperparameter names for which author do not give information.}\label{tab:ibmdp-rl1}
    \begin{tabular}{ll}
    \toprule
    \textbf{Hyperparameter} & \textbf{Values}\\
    \midrule
    Buffer size & \textcolor{green}{$10^6$} \\
    Random transitions before learning & \textcolor{green}{$10^5$} \\
    Epsilon start & 0.9, \textcolor{green}{0.5} \\
    Epsilon end & \textcolor{green}{0.05} \\
    Exploration fraction & \textcolor{green}{0.1} \\
    Optimizer & \textcolor{green}{RMSprop ($\alpha = 0.95$)}\\
    Learning rate & \textcolor{green}{$2.5\times10^{-4}$}\\
    Networks architectures & \textcolor{green}{[128, 128]}\\
    \textcolor{red}{Networks activation} & $\operatorname{tanh()}$, $\operatorname{relu()}$\\
    \bottomrule
    \end{tabular}
    \end{table}

\begin{table}[H]
    \centering
    \caption{(Modified) PPO trained on $4\times10^6$ timesteps. This gives two different instantiation of (modified) PPO. Hyperparameters not mentioned are stable-baselines3 default. In green we highlight the hyperparameters from the original paper and in red we highlight the hyperparameter names for which author do not give information.}\label{tab:ibmdp-rl2}
    \begin{tabular}{ll}
    \toprule
    \textbf{Hyperparameter} & \textbf{Values}\\
    \midrule
    Steps between each policy gradient steps & \textcolor{green}{512} \\
    Number of minibatch for policy gradient updates & \textcolor{green}{4} \\
    Networks architectures & \textcolor{green}{[64, 64]}\\
    \textcolor{red}{Networks activations} & $\operatorname{tanh()}$, $\operatorname{relu()}$\\
    \bottomrule
    \end{tabular}
    \end{table}


In tables~\ref{tab:mod-dqn} and~\ref{tab:mod-ppo} we report the top-5 hyperparameters for Modified RL baselines when learning partially observable IBMDP policies in terms of extracted decision tree policies performances in the CartPole MDP.
\begin{table}
    \centering
    \caption{Top 5 hyperparameter configurations for modified DQN + IBMDP, bold font represent the original paper hyperparameters.}\label{tab:mod-dqn}
    \label{tab:top5_results}
    \begin{tabular}{ccccccS}
    \toprule
    Rank & $q$ & Depth control & Activation & Exploration & $\zeta$ & {Mean Final Performance} \\
    \midrule
    1 & 3 & termination & $\operatorname{tanh}$ & 0.9 & 0.01 & 53 \\
    2 & 2 & termination & $\operatorname{tanh}$ & 0.5 & -0.01 & 24 \\
    \textbf{3} & \textbf{3} & \textbf{termination} & $\operatorname{tanh}$ & \textbf{0.5} & \textbf{-0.01} & \textbf{24} \\
    4 & 2 & termination & $\operatorname{tanh}$ & 0.5 & 0.01 & 23 \\
    5 & 2 & termination & $\operatorname{tanh}$ & 0.9 & -0.01 & 22 \\
    \bottomrule
    \end{tabular}
    \end{table}

    \begin{table}
        \centering
        \caption{Top 5 hyperparameter configurations for modified PPO + IBMDP, bold font represent the original paper hyperparameters.}\label{tab:mod-ppo}
        \label{tab:top5_ppo_results}
        \begin{tabular}{cccccS}
        \toprule
        Rank & $q$ & Depth Control & Activation & $\zeta$ & {Mean Final Performance} \\
        \midrule
        1 & 3 & reward & $\operatorname{relu}$ & 0.01 & 139 \\
        2 & 3 & termination & $\operatorname{relu}$ & 0.01 & 132 \\
        \textbf{3} & \textbf{3} & \textbf{reward} & $\operatorname{tanh}$ & \textbf{-0.01} & \textbf{119} \\
        4 & 3 & reward & $\operatorname{relu}$ & -0.01 & 117 \\
        5 & 3 & reward & $\operatorname{tanh}$ & 0.01 & 116 \\
        \bottomrule
        \end{tabular}
        \end{table}

    

\section{Interpretable RL objective values}\label{calcs}
\paragraph{Optimal depth-1 decision tree policy} $\pi_{\mathcal{T}_1}$ has one root node that tests $x\leq1$ (respectively $y\leq1$) and two leaf nodes $\rightarrow$ and $\downarrow$. 
    To compute $V^\pi_{\mathcal{T}_1}(\boldsymbol{o}_0)$, we compute the values of $\pi_{\mathcal{T}_1}$ in each of the possible starting states $(\boldsymbol{s}_0, \boldsymbol{o}_0), (\boldsymbol{s}_1, \boldsymbol{o}_0), (\boldsymbol{s}_2, \boldsymbol{o}_0), (\boldsymbol{s}_g, \boldsymbol{o}_0)$ and compute the expectation over those. 
    At initialization, when the base state is $\boldsymbol{s}_g = (1.5, 0.5)$, the depth-1 decision tree policy cycles between taking an information gathering action $x\leq1$ and moving down to get a positive reward for which it gets the returns:
    \begin{align*}
        V^{\pi_{\mathcal{T}_1}} (\boldsymbol{s}_g, \boldsymbol{o}_0) &= \zeta + \gamma + \gamma^2 \zeta + \gamma^3 \dots \\
        &= \overset{\infty}{\underset{t=0}\sum} \gamma^{2t} \zeta + \overset{\infty}{\underset{t=0}\sum} \gamma^{2t+1} \\
        &= \frac{\zeta + \gamma}{1 - \gamma^2}
    \end{align*}
    At initialization, in either of the base states $\boldsymbol{s}_0=(0.5,0.5)$ and $\boldsymbol{s}_2=(1.5, 1.5)$, the value of the depth-1 decision tree policy is the return when taking one information gathering action $x\leq1$, then moving right or down, then following the policy from the goal state $\boldsymbol{s}_g$:
    \begin{align*}
        V^{\pi_{\mathcal{T}_1}} (\boldsymbol{s}_0, o_0) &= \zeta + \gamma 0 + \gamma^2 V^{\pi_{\mathcal{T}_1}} (\boldsymbol{s}_g, o_0) \\
        &= \zeta + \gamma^2 V^{\pi_{\mathcal{T}_1}} (\boldsymbol{s}_g, o_0) \\
        &= V^{\pi_{\mathcal{T}_1}} (\boldsymbol{s}_2, o_0)
    \end{align*}
    Similarly, the value of the best depth-1 decision tree policy in state $\boldsymbol{s}_1=(0.5,1.5)$ is the value of taking one information gathering action then moving right to $\boldsymbol{s}_2$ then following the policy in $\boldsymbol{s}_2$:
    \begin{align*}
        V^{\pi_{\mathcal{T}_1}} (\boldsymbol{s}_1, \boldsymbol{o}_0) &= \zeta + \gamma 0 + \gamma^2 V^{\pi_{\mathcal{T}_1}} (\boldsymbol{s}_2, \boldsymbol{o}_0) \\
        &= \zeta + \gamma^2 V^{\pi_{\mathcal{T}_1}} (\boldsymbol{s}_2, o_0) \\
        &= \zeta + \gamma^2 (\zeta + \gamma^2 V^{\pi_{\mathcal{T}_1}} (\boldsymbol{s}_g, o_0)) \\
        &= \zeta + \gamma^2 \zeta + \gamma^4 V^{\pi_{\mathcal{T}_1}} (\boldsymbol{s}_g, o_0)
    \end{align*}
    Since the probability of being in any base states at initialization given that the observation is $\boldsymbol{o}_0$ is simply the probability of being in any base states at initialization, we can write:
    \begin{align*}
        V^{\pi_{\mathcal{T}_1}} (\boldsymbol{o}_0) &= \frac{1}{4} V^{\pi_{\mathcal{T}_1}} (\boldsymbol{s}_g, \boldsymbol{o}_0) + \frac{2}{4} V^{\pi_{\mathcal{T}_1}} (\boldsymbol{s}_2, \boldsymbol{o}_0) + \frac{1}{4} V^{\pi_{\mathcal{T}_1}} (\boldsymbol{s}_1, \boldsymbol{o}_0) \\
        &= \frac{1}{4} \frac{\zeta + \gamma}{1 - \gamma^2} + \frac{2}{4} (\zeta + \gamma^2 \frac{\zeta + \gamma}{1 - \gamma^2}) + \frac{1}{4} (\zeta + \gamma^2 \zeta + \gamma^4 \frac{\zeta + \gamma}{1 - \gamma^2}) \\
        &= \frac{1}{4} \frac{\zeta + \gamma}{1 - \gamma^2} + \frac{2}{4} (\frac{\zeta + \gamma ^ 3}{1-\gamma^2}) + \frac{1}{4}(\frac{\zeta+\gamma^5}{1-\gamma^2}) \\
        &= \frac{4\zeta + \gamma + 2\gamma^3 + \gamma^5}{4(1-\gamma^2)}
    \end{align*}

\paragraph{Depth-0 decision tree:} has only one leaf node that takes a single base action indefinitely.
For this type of tree the best reward achievable is to take actions that maximize the probability of reaching the objective $\rightarrow$ or $\downarrow$. In that case the objective value of such tree is:
In the goal state $G = (1, 0)$, the value of the depth-0 tree $\mathcal{T}_0$ is:
\begin{align*}
    V^{\mathcal{T}_0}_G &= 1 + \gamma + \gamma^2 + \dots \\
    &= \overset{\infty}{\underset{t=0}\sum} \gamma^t \\
    &= \frac{1}{1 - \gamma}
\end{align*}
In the state $(0, 0)$ when the policy repeats going right respectively in the state $(0, 1)$ when the policy repeats going down, the value is:
\begin{align*}
    V^{\mathcal{T}_0}_{S_0} &= 0 + \gamma V^{\mathcal{T}_0}_g \\
    &= \gamma V^{\mathcal{T}_0}_G
\end{align*}
In the other states the policy never gets positive rewards; $V^{\mathcal{T}_0}_{S_1} = V^{\mathcal{T}_0}_{S_2} = 0$. Hence:
\begin{align*}
J(\mathcal{T}_0) &= \frac{1}{4} V^{\mathcal{T}_0}_G + \frac{1}{4} V^{\mathcal{T}_0}_{S_0}+ \frac{1}{4} V^{\mathcal{T}_0}_{S_1}+ \frac{1}{4} V^{\mathcal{T}_0}_{S_2} \\
&= \frac{1}{4} V^{\mathcal{T}_0}_G + \frac{1}{4} \gamma V^{\mathcal{T}_0}_G + 0 + 0\\
&= \frac{1}{4} \frac{1}{1 - \gamma} + \frac{1}{4} \gamma \frac{1}{1 - \gamma} \\
&= \frac{1 + \gamma}{4(1 - \gamma)}
\end{align*}

\paragraph{Unbalanced depth-2 decision tree:}the unbalanced depth-2 decision tree  takes an information gathering action $x\leq0.5$ then either takes the $\downarrow$ action or takes a second information $y\leq0.5$ followed by $\rightarrow$ or $\downarrow$.
In states $G$ and $S_2$, the value of the unbalanced tree is the same as for the depth-1 tree.
In states $S_0$ and $S_1$, the policy takes two information gathering actions before taking a base action and so on:
\begin{align*}
    V^{\mathcal{T}_{u}}_{S_0} &= \zeta + \gamma \zeta + \gamma ^ 2 0 + \gamma ^ 3 V^{\mathcal{T}_1}_G
\end{align*} 
\begin{align*}
    V^{\mathcal{T}_{u}}_{S_1} &= \zeta + \gamma \zeta + \gamma ^ 2 0 + \gamma ^ 3 V^{\mathcal{T}_u}_{S_0} \\ 
    &= \zeta + \gamma \zeta + \gamma ^ 2 0 + \gamma ^ 3 (\zeta + \gamma \zeta + \gamma ^ 2 0 + \gamma ^ 3 V^{\mathcal{T}_1}_G) \\
    &= \zeta + \gamma \zeta + \gamma ^ 3 \zeta + \gamma ^ 4 \zeta + \gamma ^ 6 V^{\mathcal{T}_1}_G
\end{align*}
We get:
\begin{align*}
    J(\mathcal{T}_{u}) &= \frac{1}{4} V^{\mathcal{T}_u}_G + \frac{1}{4} V^{\mathcal{T}_u}_{S_0} + \frac{1}{4}V^{\mathcal{T}_u}_{S_1} + \frac{1}{4}V^{\mathcal{T}_u}_{S_2} \\
    &=  \frac{1}{4} V^{\mathcal{T}_1}_G + \frac{1}{4}(\zeta + \gamma \zeta + \gamma ^ 3 V^{\mathcal{T}_1}_G) + \frac{1}{4} (\zeta + \gamma \zeta + \gamma ^ 3 \zeta + \gamma ^ 4 \zeta + \gamma ^ 6 V^{\mathcal{T}_1}_G) + \frac{1}{4}V^{\mathcal{T}_1}_{S_2} \\
    &= \frac{1}{4} (\frac{\zeta + \gamma}{1-\gamma^2}) + \frac{1}{4}(\frac{\gamma\zeta + \gamma^4 + \zeta -\gamma^2\zeta}{1-\gamma^2}) + \frac{1}{4} (\zeta + \gamma \zeta + \gamma ^ 3 \zeta + \gamma ^ 4 \zeta + \gamma ^ 6 V^{\mathcal{T}_1}_G) + \frac{1}{4}V^{\mathcal{T}_1}_{S_2} \\
    &= \frac{1}{4} (\frac{\zeta + \gamma}{1-\gamma^2}) + \frac{1}{4}(\frac{\gamma\zeta + \gamma^4 + \zeta -\gamma^2\zeta}{1-\gamma^2}) + \frac{1}{4} (\frac{\zeta + \gamma\zeta -\gamma^2\zeta-\gamma^5\zeta+\gamma^6\zeta+\gamma^7}{1-\gamma^2}) + \frac{1}{4}V^{\mathcal{T}_1}_{S_2} \\
    &= \frac{1}{4} (\frac{\zeta + \gamma}{1-\gamma^2}) + \frac{1}{4}(\frac{\gamma\zeta + \gamma^4 + \zeta -\gamma^2\zeta}{1-\gamma^2}) + \frac{1}{4} (\frac{\zeta + \gamma\zeta -\gamma^2\zeta-\gamma^5\zeta+\gamma^6\zeta+\gamma^7}{1-\gamma^2}) + \frac{1}{4}(\frac{\zeta + \gamma ^ 3}{1-\gamma^2}) \\
    &= \frac{\zeta(4+2\gamma-2\gamma^2-\gamma^5+\gamma^6)+\gamma+\gamma^3+\gamma^4+\gamma^7}{4(1-\gamma^2)}
\end{align*}
\paragraph{The balanced depth-2 decision tree:}alternates in every state between taking the two available information gathering actions and then a base action.
The value of the policy in the goal state is:
\begin{align*}
    V^{\mathcal{T}_2}_{G} &= \zeta + \gamma\zeta + \gamma^2 + \gamma^3\zeta + \gamma^4\zeta + \dots \\
    &= \overset{\infty}{\underset{t=0}\sum} \gamma^{3t}\zeta + \overset{\infty}{\underset{t=0}\sum} \gamma^{3t+1}\zeta + \overset{\infty}{\underset{t=0}\sum} \gamma^{3t+2} \\
    &= \frac{\zeta}{1-\gamma^3} + \frac{\gamma\zeta}{1-\gamma^3} + \frac{\gamma^2}{1-\gamma^3}
\end{align*}
Following the same reasoning for other states we find the objective value for the depth-2 decision tree policy to be:
\begin{align*}
    J(\mathcal{T}_2) &=\frac{1}{4} V^{\mathcal{T}_2}_G + \frac{2}{4} V^{\mathcal{T}_2}_{S_2} + \frac{1}{4} V^{\mathcal{T}_2}_{S_1} \\
    &= \frac{1}{4} V^{\mathcal{T}_2}_G + \frac{2}{4}(\zeta + \gamma\zeta + \gamma^2 0 + \gamma^3V^{\mathcal{T}_2}_G) + \frac{1}{4} (\zeta+\gamma\zeta+\gamma^2 0 + \gamma^3\zeta+\gamma^4\zeta+\gamma^5 0 +\gamma^6 V^{\mathcal{T}_2}_G) \\
    &= \frac{\zeta(3+3\gamma)+\gamma^2+\gamma^5+\gamma^8}{4(1-\gamma^3)}
\end{align*}
\paragraph{Infinite tree:} we also consider the infinite tree policy that repeats an information gathering action forever and has objective: $J(\mathcal{T_{\text{inf}}}) = \frac{\zeta}{1-\gamma}$

\paragraph{Stochastic policy:} the other non-trivial policy that can be learned by solving a partially observable IBMDP is the stochastic policy that guarantees to reach $G$ after some time: fifty percent chance to do $\rightarrow$ and fifty percent chance to do $\downarrow$.
This stochastic policy has objective value:
\begin{align*}
    V^{\text{stoch}}_G &= \frac{1}{1-\gamma} \\
    V^{\text{stoch}}_{S_0} &= 0 + \frac{1}{2}\gamma V^{\text{stoch}}_G + \frac{1}{2}\gamma V^{\text{stoch}}_{S_1} \\
    V^{\text{stoch}}_{S_2} &= 0 + \frac{1}{2}\gamma V^{\text{stoch}}_G + \frac{1}{2}\gamma V^{\text{stoch}}_{S_1} = V^{\text{stoch}}_{S_0} \\
    V^{\text{stoch}}_{S_1} &= 0 + \frac{1}{2}\gamma V^{\text{stoch}}_{S_2} + \frac{1}{2}\gamma V^{\text{stoch}}_G = \frac{1}{2}\gamma V^{\text{stoch}}_{S_0} + \frac{1}{2}\gamma V^{\text{stoch}}_G
\end{align*}
Solving these equations:
\begin{align*}
    V^{\text{stoch}}_{S_1} &= \frac{1}{2}\gamma V^{\text{stoch}}_{S_0} + \frac{1}{2}\gamma V^{\text{stoch}}_G \\
    &= \frac{1}{2}\gamma (\frac{1}{2}\gamma V^{\text{stoch}}_G + \frac{1}{2}\gamma V^{\text{stoch}}_{S_1}) + \frac{1}{2}\gamma V^{\text{stoch}}_G \\
    &= \frac{1}{4}\gamma^2 V^{\text{stoch}}_G + \frac{1}{4}\gamma^2 V^{\text{stoch}}_{S_1} + \frac{1}{2}\gamma V^{\text{stoch}}_G \\
    V^{\text{stoch}}_{S_1} - \frac{1}{4}\gamma^2 V^{\text{stoch}}_{S_1} &= \frac{1}{4}\gamma^2 V^{\text{stoch}}_G + \frac{1}{2}\gamma V^{\text{stoch}}_G \\
    V^{\text{stoch}}_{S_1}(1 - \frac{1}{4}\gamma^2) &= (\frac{1}{4}\gamma^2 + \frac{1}{2}\gamma) V^{\text{stoch}}_G \\
    V^{\text{stoch}}_{S_1} &= \frac{\frac{1}{4}\gamma^2 + \frac{1}{2}\gamma}{1 - \frac{1}{4}\gamma^2} V^{\text{stoch}}_G \\
    &= \frac{\gamma(\frac{1}{4}\gamma + \frac{1}{2})}{1 - \frac{1}{4}\gamma^2} \cdot \frac{1}{1-\gamma} \\
    &= \frac{\gamma(\frac{1}{4}\gamma + \frac{1}{2})}{(1 - \frac{1}{4}\gamma^2)(1-\gamma)}
\end{align*}
\begin{align*}
    V^{\text{stoch}}_{S_0} &= \frac{1}{2}\gamma V^{\text{stoch}}_G + \frac{1}{2}\gamma V^{\text{stoch}}_{S_1} \\
    &= \frac{1}{2}\gamma \cdot \frac{1}{1-\gamma} + \frac{1}{2}\gamma \cdot \frac{\gamma(\frac{1}{4}\gamma + \frac{1}{2})}{(1 - \frac{1}{4}\gamma^2)(1-\gamma)} \\
    &= \frac{\frac{1}{2}\gamma}{1-\gamma} + \frac{\frac{1}{2}\gamma^2(\frac{1}{4}\gamma + \frac{1}{2})}{(1 - \frac{1}{4}\gamma^2)(1-\gamma)} \\
    &= \frac{\frac{1}{2}\gamma(1 - \frac{1}{4}\gamma^2) + \frac{1}{2}\gamma^2(\frac{1}{4}\gamma + \frac{1}{2})}{(1 - \frac{1}{4}\gamma^2)(1-\gamma)} \\
    &= \frac{\frac{1}{2}\gamma - \frac{1}{8}\gamma^3 + \frac{1}{8}\gamma^3 + \frac{1}{4}\gamma^2}{(1 - \frac{1}{4}\gamma^2)(1-\gamma)} \\
    &= \frac{\frac{1}{2}\gamma + \frac{1}{4}\gamma^2}{(1 - \frac{1}{4}\gamma^2)(1-\gamma)} \\
    &= \frac{\gamma(\frac{1}{2} + \frac{1}{4}\gamma)}{(1 - \frac{1}{4}\gamma^2)(1-\gamma)}
\end{align*}
\begin{align*}
    J(\mathcal{T}_{\text{stoch}}) &= \frac{1}{4}(V^{\text{stoch}}_G + V^{\text{stoch}}_{S_0} + V^{\text{stoch}}_{S_1} + V^{\text{stoch}}_{S_2}) \\
    &= \frac{1}{4}\left(\frac{1}{1-\gamma} + 2 \cdot \frac{\gamma(\frac{1}{2} + \frac{1}{4}\gamma)}{(1 - \frac{1}{4}\gamma^2)(1-\gamma)} + \frac{\gamma(\frac{1}{4}\gamma + \frac{1}{2})}{(1 - \frac{1}{4}\gamma^2)(1-\gamma)}\right) \\
    &= \frac{1}{4}\left(\frac{1}{1-\gamma} + \frac{2\gamma(\frac{1}{2} + \frac{1}{4}\gamma) + \gamma(\frac{1}{4}\gamma + \frac{1}{2})}{(1 - \frac{1}{4}\gamma^2)(1-\gamma)}\right) \\
    &= \frac{1}{4}\left(\frac{1}{1-\gamma} + \frac{\gamma + \frac{1}{2}\gamma^2 + \frac{1}{4}\gamma^2 + \frac{1}{2}\gamma}{(1 - \frac{1}{4}\gamma^2)(1-\gamma)}\right) \\
    &= \frac{1}{4}\left(\frac{1}{1-\gamma} + \frac{\frac{3}{2}\gamma + \frac{3}{4}\gamma^2}{(1 - \frac{1}{4}\gamma^2)(1-\gamma)}\right) \\
    &= \frac{1}{4}\left(\frac{1 - \frac{1}{4}\gamma^2 + \frac{3}{2}\gamma + \frac{3}{4}\gamma^2}{(1 - \frac{1}{4}\gamma^2)(1-\gamma)}\right) \\
    &= \frac{1}{4}\left(\frac{1 + \frac{3}{2}\gamma + \frac{1}{2}\gamma^2}{(1 - \frac{1}{4}\gamma^2)(1-\gamma)}\right) \\
    &= \frac{1 + \frac{3}{2}\gamma + \frac{1}{2}\gamma^2}{4(1 - \frac{1}{4}\gamma^2)(1-\gamma)}
\end{align*}

\section{Tabular RL algorithmic details for POIBMDPs}\label{sec:hp-pomdp}

\RestyleAlgo{ruled}
\SetKwComment{Comment}{}{}
\begin{algorithm}
    \KwData{POMDP $\mathcal{M}_{po} = \langle X, O, A, R, T, T_0, \Omega \rangle$, learning rates $\alpha_u,\quad \alpha_q$, exploration rate $\epsilon$}
    \KwResult{$\pi:O\rightarrow A$}
    Initialize $U(x,a) = 0$ for all $x \in X, a \in A$ \\
    Initialize $Q(o,a) = 0$ for all $o \in O, a \in A$ \\

    \For{each episode}{
        Initialize state $x_0 \sim T_0$ \\
        Initialize observation $\boldsymbol{o}_0 \sim \Omega(x_0)$ \\
        Choose action $a_0$ using $\epsilon$-greedy: $a_0 = \operatorname{argmax}_a Q(\boldsymbol{o}_0,a)$ with prob. $1-\epsilon$ \\

        \For{each step $t$}{
            Take action $a_t$, observe $r_t = R(x_t,a_t)$, $x_{t+1} \sim T(x_t,a_t)$, and $\boldsymbol{o}_{t+1} \sim \Omega(x_{t+1})$ \\
            Choose action $a_{t+1}$ using $\epsilon$-greedy: $a_{t+1} = \operatorname{argmax}_a Q(\boldsymbol{o}_{t+1},a)$ with prob. $1-\epsilon$ \\
            $y \leftarrow r + \gamma U(x_{t+1}, a_{t+1})$ \Comment{// TD target using actual next action} \\
            $U(x_t,a_t) \leftarrow (1 - \alpha_u) U(x_t, a_t) + \alpha_u y $ \\
            $Q(\boldsymbol{o}_t,a_t) \leftarrow (1 - \alpha_q) Q(\boldsymbol{o}_t, a_t) + \alpha_q y $ \\
            $x_t \leftarrow x_{t+1}$ \\
            $\boldsymbol{o}_t \leftarrow \boldsymbol{o}_{t+1}$ \\
            $a_t \leftarrow a_{t+1}$ \\
        }
    }
    $\pi(b\oldsymbol{o}) = \operatorname{argmax}_a Q(\boldsymbol{o},a)$ \Comment{// Extract greedy policy}
    \caption{Asymmetric Sarsa}\label{alg:asymsarsa}
\end{algorithm}

\RestyleAlgo{ruled}
\SetKwComment{Comment}{}{}
\begin{algorithm}
    \KwData{POMDP $\mathcal{M}_{po} = \langle X, O, A, R, T, T_0, \Omega \rangle$, learning rate $\alpha$, policy parameters $\theta$, number of trajectories $N$}
    \KwResult{Stochastic partially observable policy $\pi_\theta: O\rightarrow \Delta(A)$}
    Initialize policy parameters $\theta$ \\
    Initialize $Q(\boldsymbol{o}, a) = 0$ for all observations $o$ and actions $a$ \\
    \For{each episode}{
        \For{$i = 1$ to $N$}{
            Generate trajectory $\tau_i = (\boldsymbol{s}_0, a_0, r_0, \boldsymbol{s}_1, a_1, r_1, \ldots, \boldsymbol{s}_T)$ following $\pi_\theta$ \\
            \For{each timestep $t$ in trajectory $\tau_i$}{
                $G_t \leftarrow \sum_{k=t}^{T} \gamma^{k-t} r_k$ \Comment{// Compute return}
                Store $(\boldsymbol{o}_t, a_t, G_t)$ for later averaging
            }
        }
        \For{each unique observation-action pair $(o, a)$}{
            $Q(o, a) \leftarrow \frac{1}{|\{(\boldsymbol{o}, a)\}|} \sum_{(\boldsymbol{o}, a, G)} G$ \Comment{// Monte Carlo estimate}
        }
        \For{each observation $o$}{
            \For{each action $a$}{
                $\pi_1(a|\boldsymbol{o}) \leftarrow 1.0$ if $a = \operatorname{argmax}_{a'} Q(\boldsymbol{o}, a')$, $0.0$ otherwise \Comment{// Deterministic policy from Q-values}
                $\pi(a|\boldsymbol{o}) \leftarrow (1 - \alpha) \pi(a|\boldsymbol{o}) + \alpha \pi_1(a|\boldsymbol{o}o)$ \Comment{// Policy improvement step}
            }
        }
        Reset $Q(\boldsymbol{o}, a) = 0$ for all observations $\boldsymbol{o}$ and actions $a$ \Comment{// Reset for next episode}
    }
    \caption{JSJ algorithm. Uses Monte Carlo estimates of the average reward value functions to perform policy improvements.}\label{alg:jsj}
\end{algorithm}

\begin{table}
\small
\centering
\caption{PG Hyperparameter Space (140 combinations)}
\begin{tabular}{lll}
\toprule
\textbf{Hyperparameter} & \textbf{Values} & \textbf{Description} \\
\midrule
Learning Rate (lr) & 0.001, 0.005, 0.01, 0.05, 0.1 & Policy gradient step size \\
Entropy Regularization (tau) & -1.0, -0.1, -0.01, 0.0, 0.01, 0.1, 1.0 & Entropy regularization coefficient \\
Temperature (eps) & 0.01, 0.1, 1.0, 10 & Softmax temperature \\
Episodes per Update (n\_steps) & 20, 200, 2000 & Number of episodes per policy update \\
\bottomrule
\end{tabular}
\end{table}

\begin{table}
\small
\centering
\caption{PG-IBMDP Hyperparameter Space (140 combinations)}
\begin{tabular}{lll}
\toprule
\textbf{Hyperparameter} & \textbf{Values} & \textbf{Description} \\
\midrule
Learning Rate (lr) & 0.001, 0.005, 0.01, 0.05, 0.1 & Policy gradient step size \\
Entropy Regularization (tau) & -1.0, -0.1, -0.01, 0.0, 0.01, 0.1, 1.0 & Entropy regularization coefficient \\
Temperature (eps) & 0.01, 0.1, 1.0, 10 & Softmax temperature \\
Episodes per Update (n\_steps) & 10, 100, 1000 & Number of episodes per policy update \\
\bottomrule
\end{tabular}
\end{table}


\begin{table}
\small
\centering
\caption{QL Hyperparameter Space (192 combinations)}
\begin{tabular}{lll}
\toprule
\textbf{Hyperparameter} & \textbf{Values} & \textbf{Description} \\
\midrule
Epsilon Schedules & (0.3, 1), (0.3, 0.99), (1, 1) & Initial exploration and decrease rate \\
Epsilon Schedules & (0.1, 1), (0.1, 0.99), (0.3, 0.99) & Initial exploration and decrease rate \\
Lambda & 0.0, 0.3, 0.6, 0.9 & Eligibility trace decay \\
Learning Rate (lr\_o) & 0.001, 0.005, 0.01, 0.1 & Observation Q-learning rate \\
Optimistic & True, False & Optimistic initialization \\
\bottomrule
\end{tabular}
\end{table}

\begin{table}
\small
\centering
\caption{QL-Asym Hyperparameter Space (768 combinations)}
\begin{tabular}{lll}
\toprule
\textbf{Hyperparameter} & \textbf{Values} & \textbf{Description} \\
\midrule
Epsilon Schedules & (0.3, 1), (0.3, 0.99), (1, 1) & Initial exploration and decrease rate \\
Epsilon Schedules & (0.1, 1), (0.1, 0.99), (0.3, 0.99) & Initial exploration and decrease rate \\
Lambda & 0.0, 0.3, 0.6, 0.9 & Eligibility trace decay \\
Learning Rate (lr\_o) & 0.001, 0.005, 0.01, 0.1 & Observation Q-learning rate \\
Learning Rate (lr\_v) & 0.001, 0.005, 0.01, 0.1 & State-action Q-learning rate \\
Optimistic & True, False & Optimistic initialization \\
\bottomrule
\end{tabular}
\end{table}

\begin{table}
\small
\centering
\caption{QL-IBMDP Hyperparameter Space (192 combinations)}
\begin{tabular}{lll}
\toprule
\textbf{Hyperparameter} & \textbf{Values} & \textbf{Description} \\
\midrule
Epsilon Schedules & (0.3, 1), (0.3, 0.99), (1, 1) & Initial exploration and decrease rate \\
Epsilon Schedules & (0.1, 1), (0.1, 0.99), (0.3, 0.99) & Initial exploration and decrease rate \\
Lambda & 0.0, 0.3, 0.6, 0.9 & Eligibility trace decay \\
Learning Rate (lr\_v) & 0.001, 0.005, 0.01, 00.1 & State-action Q-learning rate \\
Optimistic & True, False & Optimistic initialization \\
\bottomrule
\end{tabular}
\end{table}

\begin{table}
\small
\centering
\caption{SARSA Hyperparameter Space (192 combinations)}
\begin{tabular}{lll}
\toprule
\textbf{Hyperparameter} & \textbf{Values} & \textbf{Description} \\
\midrule
Epsilon Schedules & (0.3, 1), (0.3, 0.99), (1, 1) & Initial exploration and decrease rate \\
Epsilon Schedules & (0.1, 1), (0.1, 0.99), (0.3, 0.99) & Initial exploration and decrease rate \\
Lambda & 0.0, 0.3, 0.6, 0.9 & Eligibility trace decay \\
Learning Rate (lr\_o) & 0.001, 0.005, 0.01, 0.1 & Observation SARSA learning rate \\
Optimistic & True, False & Optimistic initialization \\
\bottomrule
\end{tabular}
\end{table}

\begin{table}
\small
\centering
\caption{SARSA-Asym Hyperparameter Space (768 combinations)}
\begin{tabular}{lll}
\toprule
\textbf{Hyperparameter} & \textbf{Values} & \textbf{Description} \\
\midrule
Epsilon Schedules & (0.3, 1), (0.3, 0.99), (1, 1) & Initial exploration and decrease rate \\
Epsilon Schedules & (0.1, 1), (0.1, 0.99), (0.3, 0.99) & Initial exploration and decrease rate \\
Lambda & 0.0, 0.3, 0.6, 0.9 & Eligibility trace decay \\
Learning Rate (lr\_o) & 0.001, 0.005, 0.01, 0.1 & Observation SARSA learning rate \\
Learning Rate (lr\_v) & 0.001, 0.005, 0.01, 0.1 & State-action SARSA learning rate \\
Optimistic & True, False & Optimistic initialization \\
\bottomrule
\end{tabular}
\end{table}

\begin{table}
\small
\centering
\caption{SARSA-IBMDP Hyperparameter Space (192 combinations)}
\begin{tabular}{lll}
\toprule
\textbf{Hyperparameter} & \textbf{Values} & \textbf{Description} \\
\midrule
Epsilon Schedules & (0.3, 1), (0.3, 0.99), (1, 1) & Initial exploration and decrease rate \\
Epsilon Schedules & (0.1, 1), (0.1, 0.99), (0.3, 0.99) & Initial exploration and decrease rate \\
Lambda & 0.0, 0.3, 0.6, 0.9 & Eligibility trace decay \\
Learning Rate (lr\_v) & 0.001, 0.005, 0.01, 0.1 & State-action SARSA learning rate \\
Optimistic & True, False & Optimistic initialization \\
\bottomrule
\end{tabular}
\end{table}


\begin{table}
    \centering
    \begin{tabular}{|l|c|c|c|}
    \textbf{Hyperparameter} & \textbf{Asym Q-learning (10/10)} & \textbf{Asym Sarsa (10/10)} & \textbf{PG (4/10)} \\
    \toprule
    epsilon\_start & 1.0 & 1.0 & - \\
    epsilon\_decay & 0.99 & 0.99 & - \\
    batch\_size & 1 & 1 & - \\
    lambda\_ & 0.0 & 0.0 & - \\
    lr\_o & 0.01 & 0.1 & - \\
    lr\_v & 0.1 & 0.005 & - \\
    optimistic & False & False & - \\
    lr & - & - & 0.05 \\
    tau & - & - & 0.1 \\
    eps & - & - & 0.1 \\
    n\_steps & - & - & 2000 \\
    \bottomrule
    \end{tabular}
    \caption{Best hyperparameters for each algorithm on the POIBMDP problem}
    \label{tab:algorithm-hyperparameters}
    \end{table}
