\chapter{Appendix I}
\label{chap-app-pomdp}

\section{Tree value computations}
\paragraph{Depth-0 decision tree:} has only one leaf node that takes a single base action indefinitely.
For this type of tree the best reward achievable is to take actions that maximize the probability of reaching the objective $\rightarrow$ or $\downarrow$. In that case the objective value of such tree is:
In the goal state $G = (1, 0)$, the value of the depth-0 tree $\mathcal{T}_0$ is:
\begin{align*}
    V^{\mathcal{T}_0}_G &= 1 + \gamma + \gamma^2 + \dots \\
    &= \overset{\infty}{\underset{t=0}\sum} \gamma^t \\
    &= \frac{1}{1 - \gamma}
\end{align*}
In the state $(0, 0)$ when the policy repeats going right respectively in the state $(0, 1)$ when the policy repeats going down, the value is:
\begin{align*}
    V^{\mathcal{T}_0}_{S_0} &= 0 + \gamma V^{\mathcal{T}_0}_g \\
    &= \gamma V^{\mathcal{T}_0}_G
\end{align*}
In the other states the policy never gets positive rewards; $V^{\mathcal{T}_0}_{S_1} = V^{\mathcal{T}_0}_{S_2} = 0$. Hence:
\begin{align*}
J(\mathcal{T}_0) &= \frac{1}{4} V^{\mathcal{T}_0}_G + \frac{1}{4} V^{\mathcal{T}_0}_{S_0}+ \frac{1}{4} V^{\mathcal{T}_0}_{S_1}+ \frac{1}{4} V^{\mathcal{T}_0}_{S_2} \\
&= \frac{1}{4} V^{\mathcal{T}_0}_G + \frac{1}{4} \gamma V^{\mathcal{T}_0}_G + 0 + 0\\
&= \frac{1}{4} \frac{1}{1 - \gamma} + \frac{1}{4} \gamma \frac{1}{1 - \gamma} \\
&= \frac{1 + \gamma}{4(1 - \gamma)}
\end{align*}

\paragraph{Unbalanced depth-2 decision tree:}the unbalanced depth-2 decision tree  takes an information gathering action $x\leq0.5$ then either takes the $\downarrow$ action or takes a second information $y\leq0.5$ followed by $\rightarrow$ or $\downarrow$.
In states $G$ and $S_2$, the value of the unbalanced tree is the same as for the depth-1 tree.
In states $S_0$ and $S_1$, the policy takes two information gathering actions before taking a base action and so on:
\begin{align*}
    V^{\mathcal{T}_{u}}_{S_0} &= \zeta + \gamma \zeta + \gamma ^ 2 0 + \gamma ^ 3 V^{\mathcal{T}_1}_G
\end{align*} 
\begin{align*}
    V^{\mathcal{T}_{u}}_{S_1} &= \zeta + \gamma \zeta + \gamma ^ 2 0 + \gamma ^ 3 V^{\mathcal{T}_u}_{S_0} \\ 
    &= \zeta + \gamma \zeta + \gamma ^ 2 0 + \gamma ^ 3 (\zeta + \gamma \zeta + \gamma ^ 2 0 + \gamma ^ 3 V^{\mathcal{T}_1}_G) \\
    &= \zeta + \gamma \zeta + \gamma ^ 3 \zeta + \gamma ^ 4 \zeta + \gamma ^ 6 V^{\mathcal{T}_1}_G
\end{align*}
We get:
\begin{align*}
    J(\mathcal{T}_{u}) &= \frac{1}{4} V^{\mathcal{T}_u}_G + \frac{1}{4} V^{\mathcal{T}_u}_{S_0} + \frac{1}{4}V^{\mathcal{T}_u}_{S_1} + \frac{1}{4}V^{\mathcal{T}_u}_{S_2} \\
    &=  \frac{1}{4} V^{\mathcal{T}_1}_G + \frac{1}{4}(\zeta + \gamma \zeta + \gamma ^ 3 V^{\mathcal{T}_1}_G) + \frac{1}{4} (\zeta + \gamma \zeta + \gamma ^ 3 \zeta + \gamma ^ 4 \zeta + \gamma ^ 6 V^{\mathcal{T}_1}_G) + \frac{1}{4}V^{\mathcal{T}_1}_{S_2} \\
    &= \frac{1}{4} (\frac{\zeta + \gamma}{1-\gamma^2}) + \frac{1}{4}(\frac{\gamma\zeta + \gamma^4 + \zeta -\gamma^2\zeta}{1-\gamma^2}) + \frac{1}{4} (\zeta + \gamma \zeta + \gamma ^ 3 \zeta + \gamma ^ 4 \zeta + \gamma ^ 6 V^{\mathcal{T}_1}_G) + \frac{1}{4}V^{\mathcal{T}_1}_{S_2} \\
    &= \frac{1}{4} (\frac{\zeta + \gamma}{1-\gamma^2}) + \frac{1}{4}(\frac{\gamma\zeta + \gamma^4 + \zeta -\gamma^2\zeta}{1-\gamma^2}) + \frac{1}{4} (\frac{\zeta + \gamma\zeta -\gamma^2\zeta-\gamma^5\zeta+\gamma^6\zeta+\gamma^7}{1-\gamma^2}) + \frac{1}{4}V^{\mathcal{T}_1}_{S_2} \\
    &= \frac{1}{4} (\frac{\zeta + \gamma}{1-\gamma^2}) + \frac{1}{4}(\frac{\gamma\zeta + \gamma^4 + \zeta -\gamma^2\zeta}{1-\gamma^2}) + \frac{1}{4} (\frac{\zeta + \gamma\zeta -\gamma^2\zeta-\gamma^5\zeta+\gamma^6\zeta+\gamma^7}{1-\gamma^2}) + \frac{1}{4}(\frac{\zeta + \gamma ^ 3}{1-\gamma^2}) \\
    &= \frac{\zeta(4+2\gamma-2\gamma^2-\gamma^5+\gamma^6)+\gamma+\gamma^3+\gamma^4+\gamma^7}{4(1-\gamma^2)}
\end{align*}
\paragraph{The balanced depth-2 decision tree:}alternates in every state between taking the two available information gathering actions and then a base action.
The value of the policy in the goal state is:
\begin{align*}
    V^{\mathcal{T}_2}_{G} &= \zeta + \gamma\zeta + \gamma^2 + \gamma^3\zeta + \gamma^4\zeta + \dots \\
    &= \overset{\infty}{\underset{t=0}\sum} \gamma^{3t}\zeta + \overset{\infty}{\underset{t=0}\sum} \gamma^{3t+1}\zeta + \overset{\infty}{\underset{t=0}\sum} \gamma^{3t+2} \\
    &= \frac{\zeta}{1-\gamma^3} + \frac{\gamma\zeta}{1-\gamma^3} + \frac{\gamma^2}{1-\gamma^3}
\end{align*}
Following the same reasoning for other states we find the objective value for the depth-2 decision tree policy to be:
\begin{align*}
    J(\mathcal{T}_2) &=\frac{1}{4} V^{\mathcal{T}_2}_G + \frac{2}{4} V^{\mathcal{T}_2}_{S_2} + \frac{1}{4} V^{\mathcal{T}_2}_{S_1} \\
    &= \frac{1}{4} V^{\mathcal{T}_2}_G + \frac{2}{4}(\zeta + \gamma\zeta + \gamma^2 0 + \gamma^3V^{\mathcal{T}_2}_G) + \frac{1}{4} (\zeta+\gamma\zeta+\gamma^2 0 + \gamma^3\zeta+\gamma^4\zeta+\gamma^5 0 +\gamma^6 V^{\mathcal{T}_2}_G) \\
    &= \frac{\zeta(3+3\gamma)+\gamma^2+\gamma^5+\gamma^8}{4(1-\gamma^3)}
\end{align*}
\paragraph{Infinite tree:} we also consider the infinite tree policy that repeats an information gathering action forever and has objective: $J(\mathcal{T_{\text{inf}}}) = \frac{\zeta}{1-\gamma}$

\paragraph{Stochastic policy:} the other non-trivial policy that can be learned by solving a partially observable IBMDP is the stochastic policy that guarantees to reach $G$ after some time: fifty percent chance to do $\rightarrow$ and fifty percent chance to do $\downarrow$.
This stochastic policy has objective value:
\begin{align*}
    V^{\text{stoch}}_G &= \frac{1}{1-\gamma} \\
    V^{\text{stoch}}_{S_0} &= 0 + \frac{1}{2}\gamma V^{\text{stoch}}_G + \frac{1}{2}\gamma V^{\text{stoch}}_{S_1} \\
    V^{\text{stoch}}_{S_2} &= 0 + \frac{1}{2}\gamma V^{\text{stoch}}_G + \frac{1}{2}\gamma V^{\text{stoch}}_{S_1} = V^{\text{stoch}}_{S_0} \\
    V^{\text{stoch}}_{S_1} &= 0 + \frac{1}{2}\gamma V^{\text{stoch}}_{S_2} + \frac{1}{2}\gamma V^{\text{stoch}}_G = \frac{1}{2}\gamma V^{\text{stoch}}_{S_0} + \frac{1}{2}\gamma V^{\text{stoch}}_G
\end{align*}
Solving these equations:
\begin{align*}
    V^{\text{stoch}}_{S_1} &= \frac{1}{2}\gamma V^{\text{stoch}}_{S_0} + \frac{1}{2}\gamma V^{\text{stoch}}_G \\
    &= \frac{1}{2}\gamma (\frac{1}{2}\gamma V^{\text{stoch}}_G + \frac{1}{2}\gamma V^{\text{stoch}}_{S_1}) + \frac{1}{2}\gamma V^{\text{stoch}}_G \\
    &= \frac{1}{4}\gamma^2 V^{\text{stoch}}_G + \frac{1}{4}\gamma^2 V^{\text{stoch}}_{S_1} + \frac{1}{2}\gamma V^{\text{stoch}}_G \\
    V^{\text{stoch}}_{S_1} - \frac{1}{4}\gamma^2 V^{\text{stoch}}_{S_1} &= \frac{1}{4}\gamma^2 V^{\text{stoch}}_G + \frac{1}{2}\gamma V^{\text{stoch}}_G \\
    V^{\text{stoch}}_{S_1}(1 - \frac{1}{4}\gamma^2) &= (\frac{1}{4}\gamma^2 + \frac{1}{2}\gamma) V^{\text{stoch}}_G \\
    V^{\text{stoch}}_{S_1} &= \frac{\frac{1}{4}\gamma^2 + \frac{1}{2}\gamma}{1 - \frac{1}{4}\gamma^2} V^{\text{stoch}}_G \\
    &= \frac{\gamma(\frac{1}{4}\gamma + \frac{1}{2})}{1 - \frac{1}{4}\gamma^2} \cdot \frac{1}{1-\gamma} \\
    &= \frac{\gamma(\frac{1}{4}\gamma + \frac{1}{2})}{(1 - \frac{1}{4}\gamma^2)(1-\gamma)}
\end{align*}
\begin{align*}
    V^{\text{stoch}}_{S_0} &= \frac{1}{2}\gamma V^{\text{stoch}}_G + \frac{1}{2}\gamma V^{\text{stoch}}_{S_1} \\
    &= \frac{1}{2}\gamma \cdot \frac{1}{1-\gamma} + \frac{1}{2}\gamma \cdot \frac{\gamma(\frac{1}{4}\gamma + \frac{1}{2})}{(1 - \frac{1}{4}\gamma^2)(1-\gamma)} \\
    &= \frac{\frac{1}{2}\gamma}{1-\gamma} + \frac{\frac{1}{2}\gamma^2(\frac{1}{4}\gamma + \frac{1}{2})}{(1 - \frac{1}{4}\gamma^2)(1-\gamma)} \\
    &= \frac{\frac{1}{2}\gamma(1 - \frac{1}{4}\gamma^2) + \frac{1}{2}\gamma^2(\frac{1}{4}\gamma + \frac{1}{2})}{(1 - \frac{1}{4}\gamma^2)(1-\gamma)} \\
    &= \frac{\frac{1}{2}\gamma - \frac{1}{8}\gamma^3 + \frac{1}{8}\gamma^3 + \frac{1}{4}\gamma^2}{(1 - \frac{1}{4}\gamma^2)(1-\gamma)} \\
    &= \frac{\frac{1}{2}\gamma + \frac{1}{4}\gamma^2}{(1 - \frac{1}{4}\gamma^2)(1-\gamma)} \\
    &= \frac{\gamma(\frac{1}{2} + \frac{1}{4}\gamma)}{(1 - \frac{1}{4}\gamma^2)(1-\gamma)}
\end{align*}
\begin{align*}
    J(\mathcal{T}_{\text{stoch}}) &= \frac{1}{4}(V^{\text{stoch}}_G + V^{\text{stoch}}_{S_0} + V^{\text{stoch}}_{S_1} + V^{\text{stoch}}_{S_2}) \\
    &= \frac{1}{4}\left(\frac{1}{1-\gamma} + 2 \cdot \frac{\gamma(\frac{1}{2} + \frac{1}{4}\gamma)}{(1 - \frac{1}{4}\gamma^2)(1-\gamma)} + \frac{\gamma(\frac{1}{4}\gamma + \frac{1}{2})}{(1 - \frac{1}{4}\gamma^2)(1-\gamma)}\right) \\
    &= \frac{1}{4}\left(\frac{1}{1-\gamma} + \frac{2\gamma(\frac{1}{2} + \frac{1}{4}\gamma) + \gamma(\frac{1}{4}\gamma + \frac{1}{2})}{(1 - \frac{1}{4}\gamma^2)(1-\gamma)}\right) \\
    &= \frac{1}{4}\left(\frac{1}{1-\gamma} + \frac{\gamma + \frac{1}{2}\gamma^2 + \frac{1}{4}\gamma^2 + \frac{1}{2}\gamma}{(1 - \frac{1}{4}\gamma^2)(1-\gamma)}\right) \\
    &= \frac{1}{4}\left(\frac{1}{1-\gamma} + \frac{\frac{3}{2}\gamma + \frac{3}{4}\gamma^2}{(1 - \frac{1}{4}\gamma^2)(1-\gamma)}\right) \\
    &= \frac{1}{4}\left(\frac{1 - \frac{1}{4}\gamma^2 + \frac{3}{2}\gamma + \frac{3}{4}\gamma^2}{(1 - \frac{1}{4}\gamma^2)(1-\gamma)}\right) \\
    &= \frac{1}{4}\left(\frac{1 + \frac{3}{2}\gamma + \frac{1}{2}\gamma^2}{(1 - \frac{1}{4}\gamma^2)(1-\gamma)}\right) \\
    &= \frac{1 + \frac{3}{2}\gamma + \frac{1}{2}\gamma^2}{4(1 - \frac{1}{4}\gamma^2)(1-\gamma)}
\end{align*}

\section{Hyperparameters}
\begin{table}[h]
\centering
\caption{PG Hyperparameter Space (140 combinations)}
\begin{tabular}{lll}
\toprule
\textbf{Hyperparameter} & \textbf{Values} & \textbf{Description} \\
\midrule
Learning Rate (lr) & 0.001, 0.005, 0.01, 0.05, 0.1 & Policy gradient step size \\
Entropy Regularization (tau) & -1.0, -0.1, -0.01, 0.0, 0.01, 0.1, 1.0 & Entropy regularization coefficient \\
Temperature (eps) & 0.01, 0.1, 1.0, 10 & Softmax temperature \\
Episodes per Update (n\_steps) & 20, 200, 2000 & Number of episodes per policy update \\
\bottomrule
\end{tabular}
\end{table}

\begin{table}[h]
\centering
\caption{PG-IBMDP Hyperparameter Space (140 combinations)}
\begin{tabular}{lll}
\toprule
\textbf{Hyperparameter} & \textbf{Values} & \textbf{Description} \\
\midrule
Learning Rate (lr) & 0.001, 0.005, 0.01, 0.05, 0.1 & Policy gradient step size \\
Entropy Regularization (tau) & -1.0, -0.1, -0.01, 0.0, 0.01, 0.1, 1.0 & Entropy regularization coefficient \\
Temperature (eps) & 0.01, 0.1, 1.0, 10 & Softmax temperature \\
Episodes per Update (n\_steps) & 10, 100, 1000 & Number of episodes per policy update \\
\bottomrule
\end{tabular}
\end{table}


\begin{table}[h]
\centering
\caption{QL Hyperparameter Space (192 combinations)}
\begin{tabular}{lll}
\toprule
\textbf{Hyperparameter} & \textbf{Values} & \textbf{Description} \\
\midrule
Epsilon Schedules & (0.3, 1), (0.3, 0.99), (1, 1) & Initial exploration and decrease rate \\
Epsilon Schedules & (0.1, 1), (0.1, 0.99), (0.3, 0.99) & Initial exploration and decrease rate \\
Lambda & 0.0, 0.3, 0.6, 0.9 & Eligibility trace decay \\
Learning Rate (lr\_o) & 0.001, 0.005, 0.01, 0.1 & Observation Q-learning rate \\
Optimistic & True, False & Optimistic initialization \\
\bottomrule
\end{tabular}
\end{table}

\begin{table}[h]
\centering
\caption{QL-Asym Hyperparameter Space (768 combinations)}
\begin{tabular}{lll}
\toprule
\textbf{Hyperparameter} & \textbf{Values} & \textbf{Description} \\
\midrule
Epsilon Schedules & (0.3, 1), (0.3, 0.99), (1, 1) & Initial exploration and decrease rate \\
Epsilon Schedules & (0.1, 1), (0.1, 0.99), (0.3, 0.99) & Initial exploration and decrease rate \\
Lambda & 0.0, 0.3, 0.6, 0.9 & Eligibility trace decay \\
Learning Rate (lr\_o) & 0.001, 0.005, 0.01, 0.1 & Observation Q-learning rate \\
Learning Rate (lr\_v) & 0.001, 0.005, 0.01, 0.1 & State-action Q-learning rate \\
Optimistic & True, False & Optimistic initialization \\
\bottomrule
\end{tabular}
\end{table}

\begin{table}[h]
\centering
\caption{QL-IBMDP Hyperparameter Space (192 combinations)}
\begin{tabular}{lll}
\toprule
\textbf{Hyperparameter} & \textbf{Values} & \textbf{Description} \\
\midrule
Epsilon Schedules & (0.3, 1), (0.3, 0.99), (1, 1) & Initial exploration and decrease rate \\
Epsilon Schedules & (0.1, 1), (0.1, 0.99), (0.3, 0.99) & Initial exploration and decrease rate \\
Lambda & 0.0, 0.3, 0.6, 0.9 & Eligibility trace decay \\
Learning Rate (lr\_v) & 0.001, 0.005, 0.01, 00.1 & State-action Q-learning rate \\
Optimistic & True, False & Optimistic initialization \\
\bottomrule
\end{tabular}
\end{table}

\begin{table}[h]
\centering
\caption{SARSA Hyperparameter Space (192 combinations)}
\begin{tabular}{lll}
\toprule
\textbf{Hyperparameter} & \textbf{Values} & \textbf{Description} \\
\midrule
Epsilon Schedules & (0.3, 1), (0.3, 0.99), (1, 1) & Initial exploration and decrease rate \\
Epsilon Schedules & (0.1, 1), (0.1, 0.99), (0.3, 0.99) & Initial exploration and decrease rate \\
Lambda & 0.0, 0.3, 0.6, 0.9 & Eligibility trace decay \\
Learning Rate (lr\_o) & 0.001, 0.005, 0.01, 0.1 & Observation SARSA learning rate \\
Optimistic & True, False & Optimistic initialization \\
\bottomrule
\end{tabular}
\end{table}

\begin{table}[h]
\centering
\caption{SARSA-Asym Hyperparameter Space (768 combinations)}
\begin{tabular}{lll}
\toprule
\textbf{Hyperparameter} & \textbf{Values} & \textbf{Description} \\
\midrule
Epsilon Schedules & (0.3, 1), (0.3, 0.99), (1, 1) & Initial exploration and decrease rate \\
Epsilon Schedules & (0.1, 1), (0.1, 0.99), (0.3, 0.99) & Initial exploration and decrease rate \\
Lambda & 0.0, 0.3, 0.6, 0.9 & Eligibility trace decay \\
Learning Rate (lr\_o) & 0.001, 0.005, 0.01, 0.1 & Observation SARSA learning rate \\
Learning Rate (lr\_v) & 0.001, 0.005, 0.01, 0.1 & State-action SARSA learning rate \\
Optimistic & True, False & Optimistic initialization \\
\bottomrule
\end{tabular}
\end{table}

\begin{table}[h]
\centering
\caption{SARSA-IBMDP Hyperparameter Space (192 combinations)}
\begin{tabular}{lll}
\toprule
\textbf{Hyperparameter} & \textbf{Values} & \textbf{Description} \\
\midrule
Epsilon Schedules & (0.3, 1), (0.3, 0.99), (1, 1) & Initial exploration and decrease rate \\
Epsilon Schedules & (0.1, 1), (0.1, 0.99), (0.3, 0.99) & Initial exploration and decrease rate \\
Lambda & 0.0, 0.3, 0.6, 0.9 & Eligibility trace decay \\
Learning Rate (lr\_v) & 0.001, 0.005, 0.01, 0.1 & State-action SARSA learning rate \\
Optimistic & True, False & Optimistic initialization \\
\bottomrule
\end{tabular}
\end{table}


\begin{table}[h]
    \centering
    \begin{tabular}{|l|c|c|c|}
    \textbf{Hyperparameter} & \textbf{Asym Q-learning (10/10)} & \textbf{Asym Sarsa (10/10)} & \textbf{PG (4/10)} \\
    \toprule
    epsilon\_start & 1.0 & 1.0 & - \\
    epsilon\_decay & 0.99 & 0.99 & - \\
    batch\_size & 1 & 1 & - \\
    lambda\_ & 0.0 & 0.0 & - \\
    lr\_o & 0.01 & 0.1 & - \\
    lr\_v & 0.1 & 0.005 & - \\
    optimistic & False & False & - \\
    lr & - & - & 0.05 \\
    tau & - & - & 0.1 \\
    eps & - & - & 0.1 \\
    n\_steps & - & - & 2000 \\
    \bottomrule
    \end{tabular}
    \caption{Best hyperparameters for each algorithm on the POIBMDP problem}
    \label{tab:algorithm-hyperparameters}
    \end{table}
